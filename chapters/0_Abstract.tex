\chapter*{Abstract}
%https://www.scribbr.com/dissertation/abstract/

% With the invention of Bitcoin and the emergence of blockchain platforms, consortium blockchain networks came to existence. Hyperledger Fabric is such a tool to provide businesses with permissioned blockchain networks and accelerate their workflows. Additionally, business and the industry use \acrshort{iot} devices for automation, control, tracking and monitoring, but being in their early stage they introduce a notable attack surface \cite{secure-IoT}. In this thesis, we investigate the applicability of blockchains on \acrshort{iot} and we introduce a lightweight SDK, developed by Dr. Gero Dittmann \cite{gero} and Dr. Jens Jelitto \cite{jens} from IBM Zurich, to enable devices transact with a Hyperledger Fabric Network. Finally, we conclude with an analysis of the SDK with respect to performance.

% In this thesis we built the fundamentals of blockchains starting from the origin, Bitcoin. We continue with Ethereum and how it expanded on Bitcoin.  We proceed with introducing \acrlong{hlf}, an adaptation of blockchain as an idea and a network to help accelerate transactions between parties. 

% Our research question formulates around the applicability of blockchain on internet of things and 


%Problem statement
%The first sentence establishes the topic and main problem that the research will address. This problem will lead to the objective and research questions.

% Environmental non-profit organizations in the UK currently face a significant funding gap. 
With the invention of Bitcoin and the emergence of blockchain platforms, consortium blockchain networks came to existence. \acrlong{hlf} is a tool to provide businesses with permissioned blockchain networks and accelerate their workflows. Additionally, industry  uses \acrlong{iot} devices for automation, controlling, tracking and monitoring, but being in their early stage they introduce a notable attack surface \cite{secure-IoT} and they suffer from standardization. Research has shown that blockchains can be effective on the security of \acrshort{iot} devices as well as offer a common platform to create standards. This study aims to bridge \acrshort{iot} and permissioned blockchain setups with \acrlong{hlf} and provide a performance analysis of a tool\footnote{developed by Dr. Gero Dittmann \cite{gero} and Dr. Jens Jelitto \cite{jens} from IBM Zurich} that enables low-powered devices to transact with a blockchain network. Building on existing work on the applicability of \acrshort{iot} and blockchains, it asks: How can a low-powered device transact with a \acrlong{hlf} network, what are the benefits, how does it perform, can it scale down further and open the doors to microcontrollers? In this context, consortium or permissioned blockchains are defined as networks that connect known entities which collaborate and/or have competing interests.

Our approach is based on a review of literature on permissioned blockchains, IoT security, use cases and finally a performance analysis of a tool that bridges \acrshort{iot} and \acrlong{hlf}. Indeed blockchains can strengthen the security and automation of \acrshort{iot} and can be seen as a provider of trust for transacting devices. Performance analysis demonstrated an improvement of 71\% on bandwidth usage and 53\% on CPU resources, thus establishing the first step to scale down further.

% Background 
% There is a brief summary of the scholarly context to show the study's relevance.

%Research has shown that donation intention is influenced by campaign messaging strategies, and that representations of individual victims are generally more effective than appeals based on abstract concepts like climate change.



%Objective
%Next, the specific objective of the research is stated.
% This study aims to determine how environmental organizations can target fundraising campaigns to increase donations. 

% This study aims to bridge \acrshort{iot} and permissioned blockchain setups with \acrlong{hlf} and provide a performance analysis on a tool\footnote{developed by Dr. Gero Dittmann \cite{gero} and Dr. Jens Jelitto \cite{jens} from IBM Zurich} that enables low-powered devices to transact with a blockchain network.

%Research question
%The topic is narrowed down even further in the main research question, which shows exactly what the study aims to find out.

%Building on existing work on targeted fundraising, it asks: To what extent does a potential donor's social distance from climate change victims in fundraising campaigns affect their intention to make a donation?

% Building on existing work on the applicability of \acrshort{iot} and blockchains, it asks: How can a low-power device transact with a \acrshort{hlf} network, what are the benefits, how does it perform, can it further scale down and open the doors to \acrshort{mcu}s?

%Definition 
%If your abstract uses specialized terms that would be unfamiliar to the average academic reader or that have various different meanings, give a concise definition.
%In this context, social distance is defined as the extent to which people feel they are in the same social group (in-group) or another social group (out-group) in relation to climate change victims.

% In this context, consortium or permissioned blockchains are defined as networks that connect known entities that collaborate and/or have competing interests.
%----

%Methodology
%The next step is a brief description of the methods used to answer the research question.
%Based on a review of the literature on donation intention and theories of social distance, an online survey was distributed to potential donors based across the UK. Respondents were randomly divided into two conditions (large and small social distance) and asked to respond to one of two sets of fundraising material. 

% Our approach is based on a review of literature on permissioned blockchains, IoT security, use cases and finally a performance analysis of a tool that bridges \acrshort{iot} and \acrshort{hlf}. 

%Results
%The most important results are summarized.
%Analysis of the responses demonstrated that large social distance was associated with stronger donation intentions than small social distance.
% Indeed blockchains can strengthen the security and automation of \acrshort{iot} and can be seen as a provider of trust for transacting devices. Performance analysis demonstrated an improvement of 71\% on bandwidth usage and 53\% on CPU resources, thus establishing the first step to scale down further.

%Conclusions
%Finally, the study's main conclusions are stated, directly answering the research question and objective. As this research focused on a practical problem, it also includes recommendations.
%The results indicate that social distance does have an impact on donation intention. On this basis, it is recommended that environmental organizations use social distance as a key factor in designing and targeting their campaigns. Further research is needed to identify other factors that could strengthen the effectiveness of these campaigns.