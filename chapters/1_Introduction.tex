\chapter{Introduction}
\pagenumbering{arabic}

\section{Motivation} %needs integration between the work done, the need for industry and the history of blockchains

In 2008, Satoshi Nakamoto published the Bitcoin whitepaper\cite{nakamoto2008bitcoin} where he described a decentralized network that enabled electronic currency with some very peculiar characteristics. The main achievement of this network, that came into existence in early 2009, is that it solves the double spending problem by making everyone's best economic interest to support an honest network. Blockchain as cryptographically linking blocks in an append-only data structure existed long before the introduction of Bitcoin \cite{haber1990time}. Young Vitalik Buterin, inspired by the work of Satoshi Nakamoto and already contributing to the Bitcoin codebase published the whitepaper of Ethereum \cite{buterin2013ethereum}. Ethereum is a Turing complete solution contradicting to Bitcoin. With the rise of Turing complete blockchain networks, consortium blockchains came to existence. We are investigating one of the Hyperledger frameworks, the \acrlong{hlf} \cite{androulaki2018hyperledger}. Fundamentally it is a \acrfull{dlt}, which means a shared ledger in a permissioned network with competing interest players. Consortium blockchains or \acrshort{dlt}s are created to accelerate \acrshort{b2b} and \acrshort{b2b2c} interactions, at their base they are centralized-authoritarian, distributed databases, recorded in a blockchain fashion, with transaction based signed messages. 

With the surge of \acrfull{iot} devices in consumer products and in the industry the attack surface has increased and security breaches have risen \cite{gemalto-iot}. Thus, under some circumstances, distributed ledgers can be a suitable solution to mitigate specific attacks and accelerate the current workflow. We believe, it is always important to have the right tools for the job and make them easily accessible to everyone.

\section{Research Topics}
We aim to explore the following research topics in this thesis:

1. Consortium Blockchains and applications.

2. Integration of Internet of Things and Blockchains.

3. Identity of things and trusted operations.

% \section{Scope}
% This study focuses on making trustless IoT devices by enabling them to transact with a consortium blockchain, \acrfull{hlf}.

% We start by covering some basic background around blockchains while explaining Bitcoin and Ethereum. We continue by examining the transaction flow of Hyperledger Fabric and we suggest, then implement a solution to abstract it and make it available for low-powered IoT devices.

\section{Outline}
% We start at chapter 2 with an introduction to cryptographical hash functions, asymmetric cryptography and \acrfull{pki}. We continue with Bitcoin and how transactions take place and consensus is reached, we expand with Ethereum and finally make an abstraction of blockchain attributes, summarizing the key characteristics of a permissionless blockchain network. 

We start at chapter 2 with an introduction to cryptographical hash functions, asymmetric cryptography and \acrfull{pki}. We continue with Bitcoin and how transactions take place and consensus is reached and we expand on Ethereum. 

In chapter 3 we discuss about the adoption, mutation of blockchains in the industry and how they differ from public networks. We continue with Internet of Things in industry and finally we conclude by merging both topics with a case study.

In chapter 4 we dive in \acrlong{hlf}, our tool of choice. We elaborate on transactions, smart contracts and key components of this network. 

In chapter 5 we present related work done with \acrshort{iot} and blockchains, what other companies do, what academia researches, what IBM does and how we expand on that.

In chapter 6 we offer an analysis of a new tool while providing a brief background on software profiling. 

In chapter 7 we conclude and discuss about future work.